\chapter{Conclusion}

This work aims to provide a post-quantum design and evaluation for IEEE 802.1X and 802.1AE. As the foundation for such a design, multiple requirements are defined. Requirements on the symmetric parts of the protocol that arise from the implications of Grover's algorithm can be solved by increasing the key sizes for symmetric encryption by a constant factor. For components of the protocols that rely on asymmetric cryptography, a solution that involves bigger problem spaces is not a viable option due to the polynomial runtime of Shor's algorithm. Most of the requirements on the asymmetric key exchange and authentication in 802.1X can be solved by inheriting security guarantees already provided by the EAP-TLS framework, which is already used in many modern IEEE 802.1X implementations. To provide security against quantum adversaries, the implementation includes an OpenSSL fork by the Open Quantum Safe project. The fork includes the vast majority of post-quantum algorithms chosen for Round~3 of the \ac{NIST} \ac{PQ} project. Also, it includes the notion of hybrid algorithms that allows using a combination of post-quantum and classical algorithms.

Further, and due to the \ac{NIST} \ac{PQ} project's API-design, the design also achieves the notion of forward-secrecy. Another requirement focuses on performance-related aspects for the asymmetric primitives used as a replacement for the currently used classical variants. To pick a contester from the NIST PQ project suitable for a PQ design of IEEE 802.1X, an extensive evaluation of most algorithms currently employed in the project is provided. Contrary to the common understanding of post-quantum key-exchange algorithms, the evaluation not only shows that there are algorithms that perform only slightly worse when compared to classical variants but even perform on-par or even slightly better in some cases. While multiple algorithms perform orders of magnitude worse than classical \ac{ECDH} based variants, some show similar or even slightly better results than selected \ac{ECDH} and RSA-based schemes. Especially lattice-based variants like Kyber, Saber and NTRU provide a smaller overhead in terms of CPU cycles than the contester algorithms. Another interesting result from the evaluation shows that using a forward-secrecy variant for the post-quantum algorithms does not introduce a significant overhead on a single key exchange's overall runtime instead of variants where a static key is used. For signature algorithms, the interpreted results indicate a similarly positive outlook. The evaluation shows promising results for the runtime of the algorithms. While no algorithm performed better than a classical \ac{ECDH}-based variant, some algorithms perform only slightly worse when measured over the runtime of a complete authentication. As for the key-exchange algorithm, lattice-based variants show the best performance with acceptable traffic patterns. The additional traffic costs introduced by most algorithms can be problematic in scenarios where bandwidth is a limiting factor. Some of the presented algorithms require a few Kilobyte to one Megabyte of traffic for a single TLS handshake. Even in scenarios like IEEE 802.1X, where the handshake usually happens in the same \ac{LAN} infrastructure, these values can be considered problematic. On the lower end, two algorithms perform notably well. Dilithium and Falcon both need about ten Kilobyte of traffic for a complete SSL handshake. This is practical for modern Ethernet networks, but modern \acs{ECDH}-based algorithms still need a notably lower amount of traffic. The most notable result from the evaluation is that the latency between the Supplicant and Authentication Server significantly impacts the authentication's overall runtime. This is due to the request-response nature of EAP(OL) and is an interesting point for future optimizations.

A practical implementation of the design in cooperation with our partner ADVA Optical Networking is described and evaluated. This practical evaluation covers a hybrid post-quantum key exchange and authentication as a proof of concept and for a final comparison of classical vs. post-quantum variants of the protocols. For the practical evaluation, Saber is chosen for the key exchange algorithm and falcon1024 for the signature algorithm. For the classical algorithms, a pure \ac{ECDH} variant and an RSA variant with 4096-bit keyspace are used for comparison. A hybrid algorithm which combines ECDH and PQ algorithms is also evaluated. The practical evaluation can reproduce the theoretical evaluation's result and again shows a promising outlook for lattice-based post-quantum algorithms. 

\section*{Future Work}
Due to constrains in time and scope, some parts of a PQ design for MACSec couldn't be considered in this thesis and are regarded as future work:

\subsubsection*{TEAP}
    As discussed in Chapter~4, EAP-TLS is inferior to TEAP in terms of supported security claims. TEAP supports cryptography binding by introducing a RADIUS TLV field that includes a \ac{MAC} to combine the RADIUS TLV values used outside of the tunnel and the outer TLS-tunnel to the inner EAP method. While the author does not expect much difference to EAP-TLS as evaluated in this thesis, it may be interesting to evaluate TEAP and compare it to this work's results.

\subsubsection*{Adapt to Further Development of the NIST Project}
    The NIST PQ project is an ongoing process that may result in drastic changes to some involved cryptography algorithms. It is important to monitor the process further and take the results into account when considering this work. While changes to parameter selections of the algorithms may not skew the evaluation results entirely, they may lead to a reconsideration of the results. An adaption to the future development of the NIST PQ project, therefore, should be done when the final results of the third round of the evaluation are available. 

\subsubsection*{PQ-TLS}
    While a PQ EAP-TLS design profits from the modular design of TLS 1.3, the protocol still is very reliant on (EC)DH-based key exchanges. It may be valuable for future designs of TLS to consider a more generic approach that includes a generic key exchange-API and support for hybrid schemes. This may be relevant for PQ design and also benefit classical TLS implementation to a certain degree. One example includes the ``trust'' on certain ECDH curve instances. By supporting arbitrary hybrid key exchanges, it would be possible to extend the security of a TLS key exchange on multiple elliptic curve parameter selections. If one parameter selection turns out to be insecure in the future, the key exchange's overall security could still be maintained.

\subsection*{Open Quantum Safe}
Finally, the author wants to acknowledge the work provided by the Open Quantum Safe project that provided many implementations and designs used in this thesis, namely liboqs, liboqs/OpenSSL, PQFresh and much pioneer work for practical post-quantum implementations. Without them, this thesis would have been impossible in this form. Disclosure of research in the form of a public available \ac{FOSS} implementation is not self-evident and shows the importance of transparent and free software for the computer science community. The author hopes that the small contributions and the forks associated with this thesis are helpful for future research on this topic.

\endinput