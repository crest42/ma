\documentclass[fleqn,compress,utf8,aspectratio=169,t]{beamer}
% use LMU theme
\usetheme{LMU}

% This setups a lot of recommended settings and packages you usually need
% anyway. So let me take care for you and your document stays clear.

% Examples: Tables, Graphics, Listings, Hyperref...
%
\input{recommended-settings}

\usepackage[style=ieee]{biblatex}
\addbibresource{slides.bib}
\usepackage{tikz}
\usepackage{amsmath}
\usepackage[utf8]{inputenc}
\usepackage{pgf-umlsd}
\usetikzlibrary{timeline}
\usetikzlibrary{decorations.pathreplacing}

%%%%%%%%%%%%%%%%%%%%%%%%%%%%%%%%%%%%%%%%%%%%%%%%%%%%%%%%%%%%%%%%%%%%%%%%%%%%%%%%
%%                                 Customizations                             %%
%%%%%%%%%%%%%%%%%%%%%%%%%%%%%%%%%%%%%%%%%%%%%%%%%%%%%%%%%%%%%%%%%%%%%%%%%%%%%%%%

% If you want no navigation uncomment this
%\beamertemplatenavigationsymbolsempty

% setup listings
\lstset{
  basicstyle=\ttfamily\color{black},
  showstringspaces=false
}

\lstdefinestyle{highlight}{
  keywordstyle=\color{red},
  commentstyle=\color{lmu@darkgray}
}

\lstdefinestyle{basetex}{
language={[LaTeX]Tex}, 
basicstyle=\color{black!40},
keywordstyle=\color{red!40},
commentstyle=\color{lmu@darkgray!40},
moredelim=**[il][\only<+>{\color{black}\lstset{style=highlight}}]{@}
}

\lstdefinestyle{basec}{
language=C,
basicstyle=\color{black!40},
keywordstyle=\color{red!40},
commentstyle=\color{green!40},
moredelim=**[il][\only<+>{\color{black}\lstset{style=highlight}}]{@}
}

%%%%%%%%%%%%%%%%%%%%%%%%%%%%%%%%%%%%%%%%%%%%%%%%%%%%%%%%%%%%%%%%%%%%%%%%%%%%%%%%
%%                                  Title Page Data                           %%
%%%%%%%%%%%%%%%%%%%%%%%%%%%%%%%%%%%%%%%%%%%%%%%%%%%%%%%%%%%%%%%%%
% helper command to add multiple authors
\newcommand{\newauthor}[2]{
\parbox[c]{0.26\textwidth}{
{\bfseries #1} \\
{\scriptsize{\href{mailto:#2}{#2}}}
}
%{#1}
}

% Authors
\author[Lösch]{
  \newauthor{Robin Lösch}{loesch@cip.ifi.lmu.de} 
}

\date[\today]{\today}

\title{Post-Quantum Key Exchange for IEEE 802.1AE}
\subtitle{Antrittsvortrag zur Masterarbeit}

\hypersetup{
  pdftitle={Title},
  pdfauthor={Author},
  hidelinks}


%%%%%%%%%%%%%%%%%%%%%%%%%%%%%%%%%%%%%%%%%%%%%%%%%%%%%%%%%%%%%%%%%%%%%%%%%%%%%%%%
%%                                  Document                                  %%
%%%%%%%%%%%%%%%%%%%%%%%%%%%%%%%%%%%%%%%%%%%%%%%%%%%%%%%%%%%%%%%%%%%%%%%%%%%%%%%%

\begin{document}

\begin{frame}
  \titlepage
\end{frame}

%%%%%%%%%%%%%%%%%%%%%%%%%%%%%%%%%%% Overview %%%%%%%%%%%%%%%%%%%%%%%%%%%%%%%%%%%

\section{Overview}

\subsection{Motivation}

\begin{frame}{Practical Quantum Computer}
  \begin{columns}[t]
    \column{.5\textwidth}
    When to panic?
    \begin{itemize}
    \item<2-> \#Qubits to break a n-bit key
    \begin{itemize}
      \item RSA:  \(2n+2\) \cite{haner2016factoring}
      \item DLP:  \(9n+2\ln(n)\) \cite{roetteler2017quantum}
    \end{itemize}
    \item<3-> Coherency time
    \begin{itemize}
      \item Keeping the state is tricky
      \item Hard to predict
      \item Strongly depends on technology
    \end{itemize}
  \end{itemize}
  \column{.6\textwidth}
    \vspace*{-1cm}
      \begin{figure}[t]
        \centering\includegraphics<2->[trim={0cm 0 0 40px}, clip, width=1\columnwidth]{plot_line_shor_rsa.pdf}
    \end{figure}
  \end{columns}
\end{frame}

\begin{frame}{Practical Quantum Computer}
  \begin{itemize}
  \item<1-> Even if we assume a Moore-like exp growth we still got plenty of time
  \item<2-> We should use this time!
  \begin{enumerate}
    \item<3-> {Design quantum safe crypto schemes}
    \item<4-> \textbf{Implement quantum safe crypto schemes}
  \end{enumerate}
\end{itemize}
\end{frame}


\end{document}
